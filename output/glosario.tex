\documentclass[twocolumn]{article}
\usepackage[utf8]{inputenc}
\usepackage{enumitem}
\title{Glosario de Términos Ágiles}
\author{Moisés Leiva}
\date{}
\begin{document}
\maketitle
\section*{Empirismo}
\textbf{En el marco de Scrum, el empirismo es un enfoque que se basa en la obtención de conocimiento a través de experiencias prácticas y evidencia empírica. Este enfoque enfatiza la importancia de los ciclos de retroalimentación, que incluyen transparencia, revisión constante y ajustes rápidos.}
\begin{itemize}[label=\textbullet]
\item La inspección y adaptación en las reuniones diarias de Scrum.
\item La revisión del incremento en la reunión de revisión del sprint.
\item La retrospectiva del sprint para identificar mejoras en el proceso.
\end{itemize}
\section*{Épicas}
\textbf{Las épicas son historias de usuario de gran tamaño que no se pueden completar en un solo sprint. Estas historias suelen dividirse en historias de usuario más pequeñas y manejables, que se pueden implementar en un solo sprint.}
\begin{itemize}[label=\textbullet]
\item Desarrollar un sistema de autenticación para la aplicación.
\item Implementar un motor de recomendaciones personalizadas.
\item Crear una interfaz de administración para la plataforma.
\end{itemize}
\section*{Historias de usuario (User Stories)}
\textbf{En Scrum, las historias de usuario son relatos breves y precisos que describen una funcionalidad desde la óptica del usuario final. Estas historias buscan captar las necesidades y expectativas del usuario, guiando a los equipos para crear soluciones que se alineen con esos objetivos.}
\begin{itemize}[label=\textbullet]
\item Una historia de usuario podría ser: 'Como estudiante de un curso en línea, quiero recibir notificaciones de nuevas lecciones para mantenerme al día con el contenido'.Otra historia de usuario podría ser: 'Como administrador de la plataforma, quiero poder generar reportes de uso para analizar el desempeño de los cursos'.
\end{itemize}
\section*{Tareas (Tasks)}
\textbf{Las tareas son actividades específicas que deben realizarse para completar una historia de usuario. Estas actividades se dividen en subtareas más pequeñas, permitiendo a los equipos de desarrollo planificar y ejecutar el trabajo de manera eficiente.}
\begin{itemize}[label=\textbullet]
\item Diseñar la interfaz de usuario.
\item Implementar la lógica de negocio.
\item Realizar pruebas de integración.
\end{itemize}
\section*{Tablero de Tareas (Task Board)}
\textbf{El tablero de tareas es una herramienta visual que se utiliza para gestionar y monitorear el progreso de las historias de usuario y tareas en un sprint. Este tablero suele dividirse en columnas que representan los estados de las tareas, como 'Por hacer' (To Do), 'En progreso' (In Progress) y 'Terminado' (Done).También se utiliza para identificar cuellos de botella y facilitar la colaboración entre los miembros del equipo.}
\begin{itemize}[label=\textbullet]
\item Una columna 'Por hacer' para las tareas pendientes.
\item Una columna 'En progreso' para las tareas en desarrollo.
\item Una columna 'Terminado' para las tareas completadas.
\end{itemize}
\section*{INVEST}
\textbf{INVEST es un acrónimo que describe las características de una historia de usuario bien definida. Estas características son: Independiente, Negociable, Valiosa, Estimable, Pequeña y Testeable. Las historias de usuario que cumplen con estos criterios suelen ser más fáciles de entender, implementar y probar.}
\begin{itemize}[label=\textbullet]
\item Una historia de usuario que pueda ser implementada de forma independiente de otras.
\item Una historia de usuario que pueda ser negociada con el cliente para ajustar los requisitos.
\item Una historia de usuario que aporte valor al usuario final.
\end{itemize}
\section*{Planning Poker}
\textbf{El Planning Poker es una técnica utilizada en Scrum para estimar el esfuerzo y la complejidad de las historias de usuario. En esta técnica, los miembros del equipo asignan puntos de historia a cada historia, basándose en su percepción de la dificultad y el trabajo requerido.}
\begin{itemize}[label=\textbullet]
\item Asignar 1 punto a una historia de usuario muy sencilla.
\item Asignar 5 puntos a una historia de usuario de complejidad media.
\item Asignar 13 puntos a una historia de usuario muy compleja.
\end{itemize}
\section*{Kanban}
\textbf{Kanban es un marco de trabajo ágil que se centra en la visualización y optimización del flujo de trabajo. Este marco utiliza tableros Kanban para representar las tareas y actividades en diferentes columnas, lo que permite a los equipos gestionar y mejorar su productividad.}
\begin{itemize}[label=\textbullet]
\item Un tablero Kanban con columnas 'Por hacer', 'En progreso' y 'Terminado'.
\item Limitar el trabajo en progreso para evitar la sobrecarga de tareas.
\item Utilizar métricas como el lead time y el throughput para mejorar el rendimiento del equipo.
\end{itemize}
\section*{PMV (Producto Mínimo Viable)}
\textbf{El Producto Mínimo Viable es una versión simplificada de un producto que incluye solo las características esenciales para satisfacer las necesidades del usuario. Este enfoque permite a los equipos lanzar rápidamente un producto al mercado, obtener retroalimentación y validar su propuesta de valor.}
\begin{itemize}[label=\textbullet]
\item Lanzar una versión beta de una aplicación con las funciones básicas.
\item Crear un prototipo funcional para validar la idea del producto.
\item Implementar una característica clave para evaluar la aceptación del mercado.
\end{itemize}
\end{document}
